A lo largo del documento se han descrito los objetivos de este Trabajo Fin de Grado, la infraestructura hardware y software utilizada, el software desarrollado que incluye el driver MAVLinkServer en JdeRobot, y una versión nueva de la herramienta UAV Viewer para la teleoperación. Se ha detallado el diseño, la implementación del driver y de la herramienta de la teleoperación. En este capítulo se exponen las conclusiones obtenidas durante todo el proceso y los trabajos futuros que pueden continuar o mejorar al actual.

\section{Conclusiones}

El principal objetivo de este trabajo era proponer una solución para el desarrollo de nuevas aplicaciones en drones cuya placa controladora se comunique mediante el protocolo MAVLink. Teniendo en cuenta los resultados obtenidos durante las pruebas, detallados en el capítulo de 5, se puede afirmar que se ha logrado el objetivo propuesto. 

Se han cubierto todas las fases de un proyecto software, desde el análisis de requisitos y su especificación, hasta la realización de pruebas unitarias y de integración, pasando por el diseño e implementación. 

\begin{itemize}
\item El primer subobjetivo del proyecto era el desarrollo dentro de la plataforma JdeRobot del driver que permitiera el acceso a los sensores y actuadores del cuadricóptero 3DR Solo Drone. Se comenzó con un estudio de las plataformas disponibles para el manejo y adquisición de los datos de los sensores de 3DR Solo Dron, y del SDK oficial de MAVLink. La base de este proyecto se basa en una ampliación del Trabajo Fin de Grado de Jorge Cano \cite{jorgeCano}, cuya versión sirvió como base pero se decidió reprogramarla desde cero. Los principales motivos fueron que no estaban bien diferenciadas las partes de servidor y cliente, lo cual era fuente de problemas a la hora de codificar futuras mejoras. Se decidió desarrollar un envoltorio que permitiera la interacción de aplicaciones JdeRobot con el SDK oficial de MAVLink. El resultado es el driver MAVLinkServer en Python 2.7 (también se hizo una versión en Python 3.5, pero fue necesario reprogramarla en 2.7 para integrar en la versión actual de JdeRobot, por compatibilidad con ROS Kinetic). 

Una de las mayores ventajas de MAVLinkServer es el uso de interfaces de mensajes ICE, válido para interfaces de imágenes o de control del dron desde otros nodos de una manera sencilla y eficiente. La arquitectura distribuida de ICE y la facilidad para implementar componentes en casi cualquier lenguaje de programación, hacen que interactuar con el 3DR Solo Dron a través del componente MAVLinkServer sea una tarea sencilla, reduciendo la dificultad que supone trabajar con los dispositivos hardware a bajo nivel.

\item El segundo subobjetivo del proyecto era el diseño y desarrollo de una nueva herramienta de teleoperación que gobernase el movimiento del cuadricóptero. Se encontró el problema de que la interfaz ya existente se diferenciaba mucho de un control natural. Esta nueva versión acerca a gente interesada en drones a la plataforma JdeRobot de una manera más simple. La nueva herramienta se llama UAV-Viewer.py y ha sido probada y utilizada en el Trabajo Fin de Grado de Jorge Vela\cite{jorgeVela}.

UAV-Viewer utiliza las interfaces ICE que MAVLinkServer ofrece, permitiendo la teleoperación del dron y el visionado de sus sensores. Funciona con diferentes modelos de drones, siempre y cuando sus drivers ofrezcan las interfaces soportadas. La interfaz gráfica de éste componente está descrita en un fichero XML, lo que permite una rápida modificación de la estética e incluso la funcionalidad. Los distintos hilos que componen la herramienta permiten atender debidamente la interfaz de usuario sin provocar bloqueos en la aplicación. Su diseño concurrente le permite desacoplar la interfaz gráfica de usuario de los hilos comunicación con MAVLinkServer.

\item El tercer y último subobjetivo fue la validación experimental de ambos componentes desarrollados. En esta fase se realizaron muchas de pruebas, desde experimentos para justificar el correcto funcionamiento del driver y el componente UAV-Viewer.py hasta pruebas de configuración o pruebas para estudio del comportamiento del sistema a bordo del cuadricóptero que muestran el correcto funcionamiento de la solución alcanzada.

\end{itemize}

En este proyecto, además de estos tres objetivos, también se definieron una serie de requisitos que la solución tenía que cumplir.

\begin{itemize}
\item La infraestructura a desarrollar para MAVLink versión 1.0 tenía que estar integrada en JdeRobot. El github oficial de MAVLink \footnote{\url{https://github.com/mavlink/mavlink}}, ha servido como apoyo y sobre todo como guía para las primeras etapas del desarrollo. Todas las aplicaciones desarrolladas en el proyecto son componentes JdeRobot, compatibles con su versión 5.6. Están integradas en el repositorio oficial y han sido probadas por terceros.
\item El software creado debía ser eficiente computacionalmente. Todos los componentes desarrollados en el proyecto se basan en un diseño multihilo que se aprovechan de las arquitecturas modernas multicore. Además, se ha estudiado el comportamiento, la carga y la duración de los ciclos de cada hilo, lo cual ha permitido ajustar la duración de cada ciclo controlando así el tiempo que un hilo permanece ocioso.

\end{itemize}

Este es un proyecto heterogéneo y en él se han utilizado tecnologías variadas, resumidas en el capítulo 3, como por ejemplo JdeRobot, MAVLink, Python o ICE, librerias gráficas como PyQt5, etc.

Este Trabajo Fin de Grado se ha incluido en una ponencia que ha sido aceptada en el congreso CivilDron 2018 \cite{civilDron}, titulada "Programación de aplicaciones para drones con el entorno software JdeRobot". 

\section{Lecciones prácticas}
A lo largo de todas estas pruebas se aprendieron bastantes lecciones prácticas. Sin entrar en las especificaciones técnicas, el desarrollo de aplicaciones con el 3DR Solo Dron y en general con cualquier robot u otro dispositivo hardware conlleva una serie de inconvenientes. Al tratarse de un VANT el primer requisito indispensable para realizar aplicaciones sobre él es disponer de un lugar lo suficientemente amplio como para que el dron pueda volar sin que colisione con ningún objeto. Sobre todo en el comienzo del desarrollo, es altamente probable cometer errores en el código que provoquen que el cuadricóptero se estrelle, lo que trae consecuencias negativas como la rotura de algún componente del cuadricóptero, choques con objetos que se encuentren en el mismo espacio o incluso daños físicos para las personas que se encuentren cerca. En las primeras etapas del proyecto se tuvieron que reponer distintos componentes, como por ejemplo las hélices, sin que esto supusiera un gran problema. Pese a las piezas reemplazadas, el 3DR Solo Drone es una plataforma muy robusta que soportó numerosos accidentes a lo largo de este trabajo sin sufrir daño significativo.

Es importante tener en cuenta la normativa de drones vigente en cada momento, recientemente ha sido actualizada (Diciembre 2017), y en ella se especifican, con mayor detalle respecto a las anteriores, el entorno y las situaciones climáticas en las que está permitido el uso de los drones. 

El 3DR Solo Drone tiene una batería de litio de 5200mAh a 14.8V con una autonomía de vuelo de aproximadamente 30 minutos. La carga de la batería con el cargador oficial oscila entre 50 y 60 minutos. Si todo va bien, esto significa que disponemos de 30 minutos para realizar las pruebas necesarias con nuestro código. Si algo falla y no se tienen más baterías, supone que como mínimo habrá que esperar 50 minutos para reanudar las pruebas. Por esta razón es muy recomendable tener varios juegos de baterías y registrar todo el proceso que se realiza en las pruebas, desde la grabación en vídeo del vuelo del dron hasta el registro de los datos de la aplicación que nos ayuden a identificar el problema con exactitud. A veces puede resultar frustrante la corta vida de las baterías: aparte de disponer de un periodo de tiempo corto para realizar las pruebas, con el uso continuado de la batería la calidad de ésta se deteriora provocando que la autonomía disminuya o que el manejo del cuadricóptero se vea afectado por la falta de potencia necesaria: con baja batería no se comporta igual que con batería completamente cargada.

\section{Trabajos Futuros}

El desarrollo de este trabajo ha abierto las puertas a distintas ideas que pueden estudiarse y realizarse en próximos proyectos:

Una primera línea para continuar este desarrollo sería extender el soporte a otros sensores, como el soporte para GPS, ya que aumentaría en gran medida el número de aplicaciones que se pueden realizar con MAVLinkServer. 

En segundo lugar, dar soporte a cámara a bordo del 3DR Solo Drone, ya sea a través de la API de MAVLink usando una GoPro, o usando el interfaz \texttt{cameraserver} que existe actualmente en JdeRobot.

En tercer lugar, relacionado con el punto anterior, sería deseable incorporar todo el sistema en una Intel Computer Stick a bordo del dron y enlazar directamente este pequeño computador con la placa controladora usando el puerto serie para prescindir de las comunicaciones via radio y wifi con el mando físico. 

Finalmente, la integración de este Trabajo de Fin de Grado junto a otros trabajos ya realizados con el ArDrone de Parrot, podría dar lugar a nuevas aplicaciones y funcionalidades con el 3DR Solo Drone: com el recorrido de rutas en 3D, autolocalización o el aterrizaje visual. La integración de un sistema de autolocalización en interiores (SLAM) es delicada, ya que la potencia del 3DR Solo Drone hace que sea peligroso volar en interiores con espacios reducidos. Para el caso de la implementación de aterrizaje visual se puede utilizar como referencia el Trabajo de Fin de Grado de Jorge Vela\cite{jorgeVela}.