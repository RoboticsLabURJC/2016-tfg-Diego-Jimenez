\chapter*{Resumen}
%\addcontentsline{toc}{chapter}{Resumen} % si queremos que aparezca en el \'indice
\markboth{RESUMEN}{RESUMEN} % encabezado

\hspace{1cm} Es cada vez más común el uso de drones en el día a día de las personas. Se puede observar cómo han explotado en estos últimos años como un aparato de entretenimiento. Pero los UAV tienen mucha historia, se comenzaron a utilizar hace muchos años con fines bélicos, y poco a poco su desarrollo ha permitido que se utilicen en ámbitos muy diferentes, como la robótica, gracias por ejemplo al comportamiento autónomo de este.

\hspace{1cm} Durante este proyecto se ha diseñado y programado un driver con la finalidad de que cualquier dron que use el protocolo MAVLink, pueda ser pilotado mediante comandos de velocidad. Para conseguir este objetivo se han combinado los interfaces ICE que nos proporciona el entorno de JdeRobot con las librerías propias de MAVProxy, el interprete de comandos MAVLink. 


\hspace{1cm} También se ha desarrollado una aplicación que permite pilotar cualquier tipo de dron que se da soporte en la plataforma JdeRobot, como puede ser el ArDrone o el 3DR solo drone. Esta aplicación tiene un interfaz similar a un mando de radiocontrol habitual en este tipo de robots y además es totalmente funcional ya que se han realizado pruebas tanto en simulador, como con un dron real en interiores y exteriores.

\hspace{1cm} Todo este trabajo se ha realizado con un dron real, añadiendole dificultad y un mayor valor. La interacción de todos los componentes,la concurrencia y el control en velocidad del dron resalta en este trabajo fin de grado. Todo el software desarrollado y los vídeos de las pruebas están accesibles p\'ublicamente. 

