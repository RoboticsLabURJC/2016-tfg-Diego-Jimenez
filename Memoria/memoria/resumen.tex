\chapter*{Resumen}
%\addcontentsline{toc}{chapter}{Resumen} % si queremos que aparezca en el \'indice
\markboth{RESUMEN}{RESUMEN} % encabezado

\hspace{1cm} Es cada vez más común el uso de drones en el día a día de las personas. Se puede observar cómo han explotado en estos últimos años como un aparato de entretenimiento. Pero los UAV tienen mucha historia, se comenzaron a utilizar hace muchos años con fines bélicos, y poco a poco su desarrollo ha permitido que se utilicen en ámbitos muy diferentes, como la robótica, gracias por ejemplo al comportamiento autónomo de este.

\hspace{1cm} Durante este proyecto se ha diseñado y programado un driver con la finalidad de que cualquier dron que use el protocolo MAVLink pueda ser pilotado mediante comandos de velocidad. En este driver se han combinado los interfaces ICE que proporciona el entorno de JdeRobot con las librerías propias de MAVProxy, el intérprete de comandos MAVLink. También se ha desarrollado una herramienta que permite teleoperar cualquier tipo de dron soportado en JdeRobot, como el ArDrone de Parrot o el 3DR Solo Drone de 3DR. Esta aplicación tiene un interfaz gráfico similar a un mando de radiocontrol habitual 

\hspace{1cm} Se han realizado pruebas de validación experimental del driver y de la herramienta, tanto en simulador como con un dron real en interiores y exteriores. Especialmenteinteresantes han sido los experimentos con un dron real, 

\hspace{1cm} Además la interacción de todos los componentes, la concurrencia y el control en velocidad del dron resaltan en este Trabajo Fin de Grado. Todo el software desarrollado y los vídeos de las pruebas están accesibles p\'ublicamente. 

