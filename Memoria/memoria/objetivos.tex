\section{Problemas a abordar}

Los objetivos de este trabajo fin de grado es crear un driver en la plataforma software
JdeRobot para drones que utilicen como interfaz de comunicación MAVLink. Para
abordar el problema lo hemos divido en:

\begin{enumerate}
\item Preparación del hardware necesario.
\item Desarrollar driver para pilotar drones en una interfaz de nivel medio. Se comunicara con el dron usando el protocolo MavLink y se usará el interfaz de nivel medio que facilita JdeRobot llamado CMDVel.
\item Desarrollar una herramienta que permita pilotar los drones con interfaz de nivel medio en python de manera más intuitiva. Esta herramienta es una evolución del interfaz ya existente del UavViewer que facilita JdeRobot.
\item Experimentos en dron real. Conectaremos nuestro 3DR Solo tanto al driver MavLinkServer, que nos facilitara la comunicacion y a la herramienta UavViewer que nos permitira pilotar.
\end{enumerate}


\section{Requisitos}

Para abordar con éxito los puntos expuestos anteriormente debemos cubrir los siguientes requisitos:

\begin{enumerate}
\item Preparación del hardware necesario:
	\begin{enumerate}[label=\alph*]	
    \item Compra del hardware necesario:
    	\begin{itemize}
   	 	\item 3DR Solo dron.
    	\item Webcam.
    	\item Intel Computer Stick.
    	\item Batería externa.
    	\end{itemize}
    \item Instalación de JdeRobot en Intel Computer Stick.
    \item Pruebas de vuelo con todo el equipamiento a bordo del 3DR Solo y pilotaje externo.
	\end{enumerate}
\item MavLinkServer
	\begin{enumerate}[label=\alph*]	
    \item Comunicación con el dron.
    \item Administración de los modos de vuelo.
    \item Fases de despegue y aterrizaje automáticas.
    \item Envio de comandos de velocidad al dron.
	\end{enumerate}
\item UavViewerPy
	\begin{enumerate}[label=\alph*]	
    \item Comunicación con MavLinkServer.
    \item Ordenes de despegue y aterrizaje.
    \item Envio de comandos de velocidad a MavLinkServer mediante 2 joysticks.
	\end{enumerate}
\item Experimentos en dron real.
	\begin{enumerate}[label=\alph*]	
    \item Pruebas de interconexion entre todos los componentes.
    \item Ejecución de plan de pruebas:
	    \begin{itemize}
    	\item Prueba de despegue y aterrizaje.
    	\item Prueba de vuelo controlado.
    	\end{itemize}
	\end{enumerate}
\end{enumerate}

Cómo requisitos no funcionales debemos:

\begin{enumerate}
\item Ser multiplataforma.
\item Utilizar únicamente librerías de software libre.
\item Ser 100 \% compatibles con los actuales interfaces JdeRobot.
\end{enumerate}

\section{Metodología y plan de trabajo}

Durante el ciclo de vida del proyecto se han llevado a cabo reuniones semanales de seguimiento con el tutor. En ellas se evaluaban las tareas realizadas y se marcaba qué dirección tomar para la siguiente iteración o incremento. Si los puntos marcados en la anterior reunión no se habían alcanzado se ampliaba el plazo o se discutían otras vías para avanzar. En caso contrario se proponían nuevos subobjetivos.
Para apoyarnos en nuestro desarrollo hemos utilizado principalmente 4 herramientas:
\begin{itemize}
\item GitHub como forja y control de versiones. En el repositorio \url{https://github.com/RoboticsURJC-students/2016-tfg-Diego-Jimenez}
se almacenan todos los desarrollos que son objetivo de éste TFG así como ésta memoria. También se encuentran subproductos de desarrollo que han ido surgiendo como apoyo o pruebas a los desarrollos principales.
\item Contamos también con un mediawiki en JdeRobot dónde hemos actualizado periódicamente nuestros avances acompañados con explicaciones, vídeos e imágenes. \url{http://jderobot.org/Jimenez-tfg}
\item Todos los vídeos del mediawiki han sido compartidos en Youtube.
\end{itemize}


